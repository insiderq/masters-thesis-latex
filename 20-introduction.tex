\Introduction

\section{Introduction to Distributed Ledgers}
Возможно, самый большой публичный датасет данных о транзакциях. Псевдонимность. Доступность данных, структура.
Рост популярности, информация о количестве пользователей, порядках размеров данных, количестве транзакций. 

\subsection{Ethereum Blockchain structure}
Выбор пал на эфириум - адреса ниболее соответствуют идентичностям людей в отличии от модели UTXO, в которой адреса являются одноразовыми. 

account, hash, transaction, block, nonce, 
Case with Zero Values transacted. Contracts
\subsection{Distributed ledger as a graph}
From, to, value

\section{Notes on topic relevance}
Anomaly detection, anti-fraud, risk management
Track of funds. Analytics


\section{Related work on Blockchain data Analysis}

\section{Related work on Graph data Analysis}

\section{Introduction to embedding}

\section{Problem statement}
Object
Subject 
Method??