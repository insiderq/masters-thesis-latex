\Conclusion
The cornerstone of this paper is application of Machine Learning techniques, particularly \textbf{clustering}, to a \textbf{large} dataset consist of cryptocurrency transactions.

\section{Main Findings}
At first, we built undirected and weighted \textbf{graph} interpreting transactions as interactions between two long-preserving on-chain entities. Then we examine embedding of graph structural information into low-dimension feature space in unsupervised manner. 
State-of-the-Art achievements in this area are obtained with \textbf{node2vec} embedding method. 

In comparative study of \textbf{pure graph based clustering} algorithms with \textbf{node2vec + K-means} conjunction on Random Graphs we showed that \textit{the latter} is superior not only in terms of \textbf{accuracy} but also in \textbf{scalability} and, accordingly, applicability to \textbf{real-life large} networks.

Finally, node2vec embeddings for a subset of \textbf{Ethereum Transactions} data were generated. Visual representation of low-dimension vectors is given and clustering performed.

\section{Questions remained open}
Although node2vec algorithm can be \textbf{effectively distributed} and several implementations exist (Python, c++, Apache Spark), all of them use and store \textbf{complete precomputed transition matrix} effectively requiring \textbf{quadratic}\cite{DBLP:journals/corr/abs-1805-00280} memory consumption, which isn't  necessary for random walk generation. 

Real-world use case of Ethereum addresses clustering conducted lacks of \textbf{result interpretation} since the subset used for analysis \textbf{consists a low number of known addresses}. For practical use of required techniques - embedding of a \textbf{much larger} subset is required. That, in order, requires an efficient distributed implementation of \textbf{random walks generation}. 

\section{Note on Further Research}
In this paper we showed that node2vec graph embedding algorithm promises to be a \textbf{mature tool} for applying various machine learning algorithms to a real-life large networks.

By definition, representation learning technique node2vec \textbf{is aimed at preserving small relative distance} between vertices that \textbf{co-occur during random walking} over graph. This is not the first attempt to construct embedding vectors using random walk procedure as shown in \cite{DBLP:journals/corr/abs-1709-05584} and main distinct feature of node2vec is introduction of \textbf{biased second-order random walk} procedure effectively combining BFS and DFS approaches.

That is why \textbf{we believe} that the following modification of random walk procedure can show considerable results on real-life transaction network case study:
\begin{itemize}
    \item Consider reversed direction of a transaction in directed graph
    \item Consider weights of transactions proportional to value transacted 
    \item Specially treat case with data-only (0 value) transactions
    \item Consider time of transaction occurrence and \textbf{walk-out from node only along transactions that existed prior to walk-in transaction}.
\end{itemize}

That modification of walking procedure going \textit{backwards in time by incoming transactions} fed to node2vec would place nodes in a relative distance in terms of \textit{source of funds}. 

Subsequent analysis of obtained embeddings combined with \textbf{well-known addresses} participated in \textbf{security breaches or theft} leads to \textbf{very practical} applications of this method.

