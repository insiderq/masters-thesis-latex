\chapter{Ethereum Transaction Graph Analysis}\label{chap:3}

In this chapter we apply node2vec to obtain structural information from Ethereum network transaction graph. Given the large amount of data that is stored in the blockchain, we make use of the Google BigQuery service to extract only a small relevant subset of nodes and reconstruct transaction graph. After that the node2vec was used to embed this graph into a low-dimensional vector space. The space of embeddings is used to clusterize Ethereum addresses.

\section{Obtaining the Data}
We used Google's BigQuery service to obtain only interesting subset of data through querying block-chain data with SQL. Google maintains the up-to-date complete Ethereum block-chain data on their servers and provides a structural query service to allow data scientists analyze the data without the overhead of downloading gigabytes of data and keeping it up-to-date. 

SQL allows to easily fetch required data. For example, to obtain every transaction that ever happened, we may write:


\begin{lstlisting}{SQL}
SELECT from_address, to_address, value
FROM `bigquery-public-data .crypto_ethereum.transactions`
\end{lstlisting}

\section{Exploratory analysis}
% \section{Пример ссылки на литературу}

% Ссылки на литературу \cite{NLPub,Ustalov:14}.

% \begin{landscape}

% Пример альбомной ориентации.

% \end{landscape}

\section{Clustering?}
