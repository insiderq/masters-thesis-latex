\chapter{Первая глава основной части}\label{chap:1}

\section{Пример формулы}

Оценить вероятность появления слова $w_3$ после слов $w_1$ и $w_2$, идущих подряд, можно при помощи формулы \ref{eq:likelihood}:

\begin{equation}
  p(w_3|w_1,w_2) = \frac{f(w_1, w_2, w_3)}{f(w_1, w_2)},
  \label{eq:likelihood}
\end{equation}
%
где $f(w_1, w_2, w_3)$ --- частота появления триграммы $(w_1, w_2, w_3)$ в корпусе, $f(w_1, w_2)$ --- частота появления биграммы $(w_1, w_2)$.

\section{Выводы по главе~1}
