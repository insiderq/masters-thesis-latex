\begin{abstract}
    Distributed Ledger is an owner-less database consists of transactions between network participants. That data can be easily interpreted as a large graph of interactions between entities that contains a lot of node structural information and metadata. Which makes it a good target of applying Machine Learning algorithms, particularly clustering. 
    In this paper we consider node2vec - modern method of embedding of graph node's structural information into low-dimensional vector space. We conduct a comparative study of \textit{pure graph based clustering algorithms} with \textit{node2vec + K-means conjunction} and show that \textit{the latter} is superior not only in terms of accuracy of extracting clusters from SBM Random Graph but also in terms of scalability. Finally, node2vec embeddings for a subset of Ethereum Transactions data were generated. Visual representation of low-dimension vectors is given and clustering performed.
\end{abstract}

\begin{abstract}
    Блокчейн или распределенный реестр - это публичная база данных, состоящая из транзакций между участниками сети. Используя информацию об отправителе и получателе транзакции мы построили граф взаимодействий между адресами. Такой граф содержит множество структурной информации, а так же, зачастую, большое количество мета-информации о вершинах, что делает его хорошей целью для применения алгоритмов машинного обучения. В частности, рассмотрен алгоритм кластеризации вершин. 
    В этой работе мы рассматриваем современный подход применения алгоритмов машинного обучения к графовым данным используя node2vec - метод встраивания структурной информации о вершинах в низкоразмерное векторное пространство. Мы проводим сравнительный анализ между алгоритмами кластеризации, основанными на \textit{анализе графов, как матрицы смежности} и связки \textit{node2vec + K-means} для той же задачи. Анализ показал превосходство этой связки над стандартными алгоритмами в задаче восстановления информации о кластерах в случайном графе модели SBM. Кроме того, мы получили низкоразмерное представление вершин подграфа транзакций в блокчейне Ethereum, привели его визуальное описание и провели кластеризацию.
\end{abstract}